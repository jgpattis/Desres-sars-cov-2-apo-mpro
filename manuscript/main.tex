% \documentclass[lineno,twocolumn,endfloat,biblatex]{biophys-new}
\documentclass{biophys-new}
\usepackage[utf8]{inputenc}
\usepackage{graphicx}
\usepackage[colorlinks,allcolors=cyan!70!black]{hyperref}

% Uncomment if using biblatex
% \addbibresource{sample.bib}

\usepackage{lipsum}

\title{New Biophysical Journal Template}
\runningtitle{Biophysical Journal Template} %% For page header

\author[1,]{Jason G. Pattis}
\author[1,*]{Vincent A. Voelz}
\runningauthor{Author1 and Author2} %% For page header

\affil[1]{Temple University, Philadelphia, Pennsylvania 19122, USA}


\corrauthor[*]{abx@xyz.edu}

% \papertype{Letters}
\papertype{Article}
% \papertype{Computational Tools}


\begin{document}

\begin{frontmatter}

\begin{abstract}
Each 
\end{abstract}

\begin{sigstatement}
no more than 120 words.
\end{sigstatement}
\end{frontmatter}

\section*{Outline}

\begin{enumerate}
    \item What are the rare event motions which occur in the main protease dimer?
    \begin{itemize}
        \item the slowest motion is the folding of the linker helix (residue 188 to 194) this motion is IC 1 and IC 2
        \item This helix is also found in the infectious bronchitis virus (IBV) crystal structure when IBV is in its monomeric state
        \item the second slowest motion (IC 3) is an N-terminal loop motion
        \begin{itemize}
            \color{blue}
            \item need to confirm that IC 3 is a N-terminal loop motion
        \end{itemize}
        \item These motions are not symmetric (they only occur in one monomer 
        \item Are these motions connected to active site dynamics and active site opening?
        \begin{itemize}
            \item Mutual information would suggest that many residues far from the active site have correlated motions with the active site
            \color{blue}
            \item look at active site pocket volume
            \item look at 145 to 41 distance as proxy for open to substrate
            \item look at 3 10 helix formation (139-141) as proxy for active
        \end{itemize}
    
    \end{itemize}
    \item How often do these motions occur?
    \begin{itemize}
        \item The slowest motions occur only once in the 100 us trajectory
    \end{itemize}
    \item Can people dock to the different conformations seen in this simulation?
    \begin{itemize}
        \item Yes cluster centers and populations well be available online for ensemble docking
        \begin{itemize}
            \color{blue}
            \item cluster centers will have to be pushed to osf or similar because they are to big for github
        \end{itemize}
    \end{itemize}
    \item Are there correlated motions creating an allosteric network? Can these allosteric regions affect the active site and the activity of the enzyme
    \begin{itemize}
        \item Mutual information analysis reveals correlated motions within the active site and residues 232 to 238
        \begin{itemize}
            \color{blue}
            \item Is there a pocket near 232 to 238
        \end{itemize}
    \end{itemize}
    \item Is there evidence that only one monomer is active at a time?
     \begin{itemize}
        \item look for alternating:
        \begin{itemize}
            \color{blue}
            \item active pocket volume
            \item res 41 to 145 distance
            \item 3 10 helix formation (139-141)
            \item packing between His163 and Phe140 may transfer conformational change to orhter monomer
        \end{itemize}
    \end{itemize}
\end{enumerate}


\section*{Introduction}

Covid-19 has been a massive public health threat all around the globe. Covid-19 is caused by the Sars-Cov-2 virus, a member of the Coronavirus family of viruses. Therapeutics are desperately needed. One promising target for drug development is Sars-Cov-2's 3CL-like protease also called the main protease or $M_{pro}$ made by the NSP5 gene. The $M_{pro}$ is responsible for cutting the polyproteins, an essential step the the virus' life cycle. One $M_{pro}$ inhibitor has entered phase I clinical trials and large efforts to find other inhibitors are underway.

The first crystal structure of the $M_{pro}$ became available in February \cite{Zhang409}, both an APO form and bound to a substrate like inhibitor called N3. Understanding the dynamics of this protein would be a massive benefit to drug discovery efforts. Dynamics of the active sight can inform how current coumpounds can be extended to make greater contact with the protein and improve affinity. Dynamics of the full protein provide insights into additional druggable pockets and how they are allostericly coupled to the active site.

There has also been some evidence that in the dimer only one monomer is active at a time. Structural porxies for activity can be looked for as well as a mode of allosteric communication between active sites.

Thanks for using Overleaf to write your article. Your introduction goes here! Some examples of commonly used commands and features are listed below, to help you get started. Leave a blank line between blocks of text to start a new paragraph---use \verb|\\| for separating tabular rows or hard line-breaks only. Abbreviations should be defined in the text at first mention.

Please also take note of the \verb|\section*{...}| titles in this template: they are the required sections in a regular research Article manuscript. 

In particular, the main text of regular Articles and Computational Tools manuscripts must be structured with the following sections: \textbf{Introduction}, \textbf{Materials and Methods}, \textbf{Results}, \textbf{Discussion (or Results and Discussion)}, \textbf{Conclusion}.

Theoretical manuscripts may include just a \textbf{Methods} section and do not require \textbf{Materials}.

No particular organization structure is required for Letters.

If your manuscript is accepted, the Biophysical production team will re-format the references for publication. \emph{It is not necessary to format the reference list yourself to mirror the final published form.}

\section*{Materials and Methods}

Capitalize trade names and give manufacturers' full names and addresses (city and state). 


\subsection*{Sectioning commands}

Use \verb|\section*{...}| and \verb|\subsection*{...}| to create first- and second-level headings. Sed ut perspiciatis unde omnis iste natus error sit voluptatem accusantium doloremque laudantium, totam rem aperiam, eaque ipsa quae ab illo inventore veritatis et quasi architecto beatae vitae dicta sunt explicabo. 

\subsection*{Figures and Tables}

Use the table and tabular commands for basic tables --- see Table \ref{tab:widgets}, for example. \href{http://tablesgenerator.com}{TablesGenerator.com} is a handy tool for designing tables and generating the \LaTeX{} \texttt{tabular} code, which you can copy and paste into your article here.

You can upload a figure (JPG, PNG or PDF) using the PROJECT menu (Files\ldots > Add files). To include it in your document, use the \verb|graphicx| package and the \verb|\includegraphics| command as in the code for Figure \ref{fig:view}. 

In addition, you can use \verb|\ref{...}| and \verb|\label{...}| commands for cross-references.

\begin{table}[hbt!]
\caption{An example table}
\label{tab:widgets}
\centering

\begin{threeparttable}

\begin{tabular}{c l r}
\hline
Code & Item & Quantity \\\hline
W1 & Widgets\tnote{a} & 42 \\
G35 & Gadgets & 13\tnote{b} \\
\hline
\end{tabular}

\begin{tablenotes}
\item[a] This is a table note.
\item[b] This is another table note.
\end{tablenotes}

\end{threeparttable}

\end{table}

\begin{figure}[hbt!]
\centering
\includegraphics[width=0.6\linewidth]{example-image}
\caption{A figure example.}
\label{fig:view}

\end{figure}

\section*{Results}

\LaTeX{} is great at typesetting mathematics:

Let $X_1, X_2, \ldots, X_n$ be a sequence of independent and identically distributed random variables with $\text{E}[X_i] = \mu$ and $\text{Var}[X_i] = \sigma^2 < \infty$, and let
\begin{equation}
\label{eq:CLT}
S_n = \frac{X_1 + X_2 + \cdots + X_n}{n}
      = \frac{1}{n}\sum_{i}^{n} X_i
\end{equation}
denote their mean. Then as $n$ approaches infinity, the random variables $\sqrt{n}(S_n - \mu)$ converge in distribution to a normal $\mathcal{N}(0, \sigma^2)$. Thus concludes the explanation about Eq.~\ref{eq:CLT}.


You can make lists with automatic numbering \dots

\begin{enumerate}
\item Like this,
\item and like this.
\end{enumerate}

\dots or bullet points \dots

\begin{itemize} 
\item Like this,
\item and like this.
\end{itemize}

\dots or with words and descriptions \dots

\begin{description}
\item[Word] Definition
\item[Concept] Explanation
\item[Idea] Text
\end{description}

An example quotation:

\begin{quote}
Lorem ipsum dolor sit amet, consectetur adipiscing elit, sed do eiusmod tempor incididunt ut labore et dolore magna aliqua. Ut enim ad minim veniam, quis nostrud exercitation ullamco laboris nisi ut aliquip ex ea commodo consequat.
\end{quote}


\section*{Discussion}

\LaTeX{} formats citations and references automatically using the bibliography records in your .bib file, which you can edit via the project menu. Use the \verb|\cite| command to insert a citation, like this: \cite{Chen_Nicholson00} Multiple citations can be given as \cite{Stiles_Bartol01,el-Kareh_etal93,Callaghan91}. You can use either BibTeX or biblatex: see the following subsections.

If your manuscript is accepted, the Biophysical production team will re-format the references for publication. \emph{It is not necessary to format the reference list yourself to mirror the final published form.}

\subsection*{Using bibtex} 
This is the default. Specify your \texttt{.bib} file with \verb|\bibliography{sample}| (the extension is unnecessary) near the end of your manuscript, where you want the references list to appear.

\subsection*{Using biblatex} 
Pass the \texttt{biblatex} option to the \verb|\documentclass| declaration, then specify your \texttt{.bib} file name in the \emph{preamble}: \verb|\addbibresources{sample.bib}| (the extension is necessary). Write \verb|\printbibliography| near the end of your manuscript where you want the references to appear.

\section*{Conclusion}

Sed ut perspiciatis unde omnis iste natus error sit voluptatem accusantium doloremque laudantium, totam rem aperiam, eaque ipsa quae ab illo inventore veritatis et quasi architecto beatae vitae dicta sunt explicabo. 

\section*{Author Contributions}

Author1 designed the research. Author2 carried out all simulations, analyzed the data. Author1 and Author2 wrote the article. 

\section*{Acknowledgments}

We thank G. Harrison, B. Harper, and J. Doe for their help.

% Uncomment if using bibtex (default)
\bibliography{sample}

% Uncomment if using biblatex
% \printbibliography

\section*{Supplementary Material}

An online supplement to this article can be found by visiting BJ Online at \url{http://www.biophysj.org}.

\end{document}
